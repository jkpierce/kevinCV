%%%%%%%%%%%%%%%%%%%%%%%%%%%%%%%%%%%%%%%%%
% Long Professional Curriculum Vitae
% LaTeX Template
% Version 1.1 (9/12/12)
%
% This template has been downloaded from:
% http://www.latextemplates.com
%
% Original author:
% Rensselaer Polytechnic Institute (http://www.rpi.edu/dept/arc/training/latex/resumes/)
%
% Important note:
% This template requires the res.cls file to be in the same directory as the
% .tex file. The res.cls file provides the resume style used for structuring the
% document.
%
%%%%%%%%%%%%%%%%%%%%%%%%%%%%%%%%%%%%%%%%%

%----------------------------------------------------------------------------------------
%	PACKAGES AND OTHER DOCUMENT CONFIGURATIONS
%----------------------------------------------------------------------------------------

\documentclass[10pt]{res} % Use the res.cls style, the font size can be changed to 11pt or 12pt here

\usepackage{helvet} % Default font is the helvetica postscript font
%\usepackage{newcent} % To change the default font to the new century schoolbook postscript font uncomment this line and comment the one above

\newsectionwidth{0pt} % Stops section indenting

\begin{document}
%----------------------------------------------------------------------------------------
%	YOUR NAME AND ADDRESS(ES) SECTION
%----------------------------------------------------------------------------------------

\name{Kevin Pierce\\ \\} % Your name at the top

% If you don't want one of the addresses, simply remove all the text in the first or second \address{} bracket

\address{\bf Department of Geography \\ University of British Columbia \\ 1984 West Mall, Vancouver, BC \\ V6T 1Z2, Canada
 } % Your address 1

\address{} % Your address 2

%----------------------------------------------------------------------------------------

\begin{resume}


%----------------------------------------------------------------------------------------
%	EDUCATION SECTION
%----------------------------------------------------------------------------------------

 

\section{\centerline{SUMMARY}} 
\vspace{8pt} % Gap between title and text
I am a recent PhD graduate from the department of Geography at the University of British Columbia in Vancouver, Canada. In my research I apply statistical and computational physics techniques to model landscape evolution and sediment transport in river channels, and I conduct laboratory experiments to guide model development and quantify the relevant processes. My educational background is in physics, with a special emphasis on the role of variability in physical systems.
Apart from research, the majority of my work experience is in education, with a special emphasis on developing course content using emerging educational technologies.
My goal is to continue advancing through a career as an Earth Science researcher and educator.

\vspace{0.2in} % Some whitespace between sections
\section{\centerline{EDUCATION}} 

\vspace{8pt} % Gap between title and text

{\sl Doctor of Philosophy}, 
Physical Geography \\ 
University of British Columbia, Vancouver, Canada \hfill (Fall 2016 -- Fall 2021) \\ 
THESIS: \textbf{The stochastic movements of individual streambed grains} \\
 
{\sl Master of Science}, Condensed Matter Physics. \\ 
University of British Columbia, Vancouver, Canada \hfill (Fall 2014 -- Spring 2016) \\
THESIS: \textbf{Magnetic structure of chiral graphene nanoribbons} \\


{\sl Bachelor of Science}, Physics. Minor in mathematics. \\ 
West Virginia University, Morgantown, WV, USA \hfill (Fall 2010 – Spring 2013) \\
THESIS: \textbf{Physics of random phenomena} \\
AWARDS: Outstanding Physics Senior (2013) \\

{\sl Associate of Science, Physics.}\\ 
Southern West Virginia Community College, Logan, WV, USA \hfill (Spring 2008 -- Spring 2010)\\


%----------------------------------------------------------------------------------------


\vspace{0.2in} % Some whitespace between sections
%----------------------------------------------------------------------------------------
%	PROFESSIONAL EXPERIENCE SECTION
%----------------------------------------------------------------------------------------

\section{\centerline{PUBLICATIONS}} 

\vspace{15pt} % Gap between title and text
\begin{enumerate}
	\item[] \textbf{Pierce, K.}, Hassan, M. A., \& Ferreira, R. M. L. (2021). Mechanistic-stochastic derivation of the bed load sediment flux. In Preparation.	
	\item[] \textbf{Pierce, K.}, Hassan, M. A., \& Ferreira, R. M. L. (2021). Collisional Langevin model of bedload sediment velocity distributions. In Preparation.
	\item[] \textbf{Pierce, K.}, \& Hassan, M. A. (2020). Back to Einstein: Burial‐Induced Three‐Range Diffusion in Fluvial Sediment Transport. Geophysical Research Letters, 47 (15), 1-10. doi: 10.1029/2020GL087440
	\item[] \textbf{Pierce, K.}, \& Hassan, M. A. (2020). Joint Stochastic Bedload Transport and Bed
	Elevation Model: Variance Regulation and Power Law Rests. Journal of Geophysical Research: Earth Surface, 125 (4), 1–15. doi: 10.1029/2019JF005259
	\item[] Rowley J. D., \textbf{Pierce K.}, Brant A. T., Halliburton L. E., Giles N. C., Schunemann P. G., and Bristow A. D. (2012). Broadband terahertz pulse emission from $ZnGeP_\textsubscript{2}$. Optics Letters, 37 (5),  788-790. doi: 10.1364/OL.37.000788
\end{enumerate}

%----------------------------------------------------------------------------------------

\vspace{0.2in} % Some whitespace between sections


\section{\centerline{CONFERENCE PRESENTATIONS}} 

\vspace{15pt} % Gap between title and text
\begin{enumerate}
	
	\item[] \textbf{Pierce, K.}, \& Hassan, M. A. (2021). Mechanistic-stochastic description of the bedload sediment flux, AGU General Assembly Conference Abstracts.
	
	\item[] \textbf{Pierce, K.}, \& Hassan, M. A. (2020). Collisional Langevin approach to bed load sediment velocity distributions, EGU General Assembly Conference Abstracts, EGU21-8148
	
	\item[] \textbf{Pierce, K.}, \& Hassan, M. A. (2020). Particle shape dictates critical shear stresses for sediment motion, AGU Fall Meeting Abstracts, EP013-0007
	
	\item[] Ferreira, R. M. L., Aleixo, R. F., Ricardo, A. M., \textbf{Pierce, K.}, Hassan, M. A. (2020) Turbulence in open-channel flows over mobile beds of high hydraulic conductivity, AGU Fall Meeting Abstracts, EP003-0002
	
	\item[] \textbf{Pierce, K.}, Hassan, M.A. (2019). Back to Einstein: how to include trapping processes in fluvial diffusion models?, AGU Fall Meeting Abstracts, EP51B-02
	
	\item[] \textbf{Pierce, K.}, Ferreira, R. M. L., Hassan, M. A. (2019). Three-dimensional resolution of bedload transport with binocular computer vision, AGU Fall Meeting Abstracts 2019, EP51F-2185
	
	\item[] \textbf{Pierce, K.}, Affleck, I. (2016). Edge magnetism of chiral graphene nanoribbons, Cifar Conference in Quantumm Materials. 
\end{enumerate}

%----------------------------------------------------------------------------------------

\vspace{0.2in} % Some whitespace between sections



\section{\centerline{PROFESSIONAL EXPERIENCE}} 
\vspace{8pt} % Gap between title and text

{Varsity Teams Academic Coach: \sl UBC Athletics} \hfill (Fall 2015 -- Fall 2016) \\
\begin{itemize} \itemsep -2pt % Reduce space between items
\item Conducted one-on-one and group tutorial lessons on calculus, chemistry, physics, and statistics for at-risk student athletes.
\item Mentored students in problem solving and academic goal setting.
\item Prepared and distributed problem sets and example problems to students to build student familiarity and experience with exam material.
\item Delivered academic skills workshops to groups of student athletes on topics ranging from presentation skills to time management.
\end{itemize}

{Online Course Content Developer: \sl UBC Geography Department} \hfill (Spring 2020 -- Summer 2021) \\
\begin{itemize} \itemsep -2pt % Reduce space between items
	\item Produced a 45 minute video documentary of a geomorphology field trip in collaboration with a professor and an audio-visual specialist for use in a 4th year undergraduate fluvial geomorphology course. This film educates the viewer on processes relevant to mountain river channel stability and dynamics.
	\item Developed five laboratory assignments for a first year undergraduate geomorphology course. A virtual field trip to the Tacoma watershed of Mount Rainier is presented to students using Tapestry, while datasets and assignments from the watershed are provided for students in Canvas. Using these datasets, students analyze hillslope stability, volcanic processes, river morphology, climate change, and glacial processes in a real watershed.
	\item Delivered several video explanations of ecological ideas for a first year geography course on climate and ecosystems. These videos were produced on site to provide students a virtual tour of the relevant ecosystems.
	\item Constructed five laboratory assignments for students in a third year undergraduate "Statistics in Geography" course. Students obtain the assignment packet from Github. Students conduct inferential statistics analyses on large datasets using R in Jupyter notebooks hosted on a Cloud Computing server.  Students make conclusions on real-world pollution, ecology, climate, and social problems during the assignments as they learn R programming and statistical analysis techniques.
\end{itemize}


{Teaching Assistant: \sl UBC \& WVU} \hfill (Fall 2012 -- Summer 2021) \\
\begin{itemize} \itemsep -2pt % Reduce space between items
	\item Produced assignments and exams, marked student work, delivered lectures, conducted tutorials, performed classroom demonstrations, hosted office hours, facilitated group-work sessions, led field trips, managed educational technology tools (Tapestry, Canvas, Blackboard, Github, Syzygy, Zoom, Slack), and calculated student marks for \textbf{20 courses} totalling more than \textbf{3600 hours} of contracted hourly work. Topics spanned all levels of physics and geography courses from 1st year to graduate level.
	\item Received the UBC Geography "Outstanding TA" award in 2020 in recognition of my work in moving the department courses online during the pandemic.
	\item Completed information security, inclusive teaching, classroom climate, and workplace violence prevention training workshops as requirements for these roles.
\end{itemize}
 
 \vspace{0.2in} % Some whitespace between sections


%----------------------------------------------------------------------------------------
%	COMPUTER SKILLS SECTION
%----------------------------------------------------------------------------------------

\section{\centerline{COMPUTER SKILLS}}

\vspace{8pt} % Gap between title and text

Experienced in Python, R, Matlab, and Bash. Intermediate ability in C and C++. Special proficiency in statistical analysis, Monte Carlo simulation, and image analysis problems. Broad capabilities in machine learning and many-particle physics simulations. Well-acquainted with high-performance computing, working on compute clusters, and job submissions using SLURM. Linux enthusiast with strong problem-solving abilities.

%----------------------------------------------------------------------------------------

\vspace{0.2in} % Some whitespace between sections

\section{\centerline{WORKSHOPS AND OUTREACH}}
\vspace{8pt} % Gap between title and text

{Earth Surface Processes Institute: \sl University of Colorado, Boulder} \hfill (Summer 2021) \\
\begin{itemize} \itemsep -2pt % Reduce space between items
	\item Completed an eight day immersive workshop on Earth Science modeling. Topics included numerical methods, best programming practices, open source software development, collaborative coding and version control, Landlab and pymt, high performance computing, and model uncertainty quantification. 
	\item Produced a final course project which is now hosted as an example on the CSDMS website. This project calculated the impact of wildfires on sediment delivery and landscape evolution in watersheds over a millennial timescale. 
\end{itemize}



{Instructional Skills Workshop: \sl UBC} \hfill (Summer 2020) \\
\begin{itemize} \itemsep -2pt % Reduce space between items
	\item Completed an three day workshop on teaching models, strategies to foster active learning in students, inclusive teaching, lesson planning, and applying emerging technologies in interactive lessons.
	\item Delivered three peer-reviewed lessons in the course of this workshop aimed at identifying and improving on weak points in my own teaching methods.
\end{itemize}


{Science 101: \sl UBC} \hfill (Summer 2021) \\
\begin{itemize} \itemsep -2pt % Reduce space between items
	\item Volunteered as an educator for Science 101, a community outreach program at UBC designed to introduce traditionally-excluded communities and older adults to scientific ideas.
	\item Mainly I led discussions on lectures from invited professors, although I also delivered lessons on plate tectonics and volcanism, identifying reliable sources, and distinguishing different plant species. 
\end{itemize}

{Books for Me!: Vancouver, BC} \hfill (Fall 2017 -- Fall 2019) \\
\begin{itemize} \itemsep -2pt % Reduce space between items
	\item Volunteered at a non-profit to encourage literacy among disadvantaged children in Vancouver's downtown East side.
	\item Joined a team to secure book donations, then host events at elementary schools to distribute books to children for free.
\end{itemize}


\vspace{0.2in} % Some whitespace between sections




\end{resume} 
\end{document}